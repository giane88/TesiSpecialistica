\newpage
\chapter*{Sommario}

\addcontentsline{toc}{chapter}{Sommario}

Questa tesi è stata sviluppata in collaborazione con \emph{LIS S.p.a.}, vigilanza privata che si distingue per i tempi di intervento ridotti e la possibilità di gestione degli impianti da remoto. Queste caratteristiche distinguono LIS già dai primi anni di attività quando ancora la ricezione degli allarmi e la telegestione avveniva tramite linee telefoniche tradizionali.
Negli ultimi anni, tuttavia, la scomparsa delle linee telefoniche tradizionali in favore di quelle VoIP e di connessioni in fibra ottica hanno creato diversi problemi alla normale ricezione degli allarmi e ai meccanismi di telegestione.
Lo scopo di questa tesi è quello di progettare e realizzare un sistema di ricezione allarmi e telegestione unificato che sia tuttavia modulare, che utilizzi nuovi canali di comunicazione e che possa essere integrato con il software preesistente.
Nel seguito vedremo come questo lavoro ci ha permesso di ridurre i tempi di integrazione di nuovi metodi di ricezione e telegestione e di aumentare conseguentemente il numero di centrali connesse e gestite dalla centrale operativa.
