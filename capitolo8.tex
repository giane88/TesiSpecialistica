\chapter{Conclusione}
\label{capitolo8}
\thispagestyle{empty}
In questa sezione trarremo alcune conclusioni sul lavoro svolto analizzando gli obiettivi raggiunti, quelli da migliorare e gli sviluppi futuri.
\section{Obiettivi raggiunti}
Lo scopo del nostro lavoro era quello di dare una struttura al software aziendale, migliorando in particolare la velocità di sviluppo di nuove funzioni e l'integrazione di nuove centrali sia dal lato della ricezione degli allarmi sia da quello della telegestione. Oltre a questo uno dei nostri obiettivi, anche se raggiunto parzialmente era quello di aggiornare tutto il reparto software utilizzando le ultime tecnologie disponibili. Sotto questo aspetto l'utilizzo di un framework per la gestione delle connessioni e l'utilizzo del linguaggio C++ aggiornato allo standard del 2011 tramite l'ausilio del compilatore gcc-4.8 ci hanno permesso notevoli migliorie, sia per quanto riguarda le prestazioni e l'affidabilità sia per quanto riguarda la velocità di sviluppo.\\
Un altro aspetto da non trascurare è l'utilizzo di versioni del sistema operativo aggiornate in modo da poter sfruttare appieno le ultime tecnologie come i multiprocessori o la tecnologia LVM per la gestione dei dischi in ambiente virtuale.\\
Pur essendo il settore della sicurezza privata non molto veloce nell'adozione delle nuove tecnologie riteniamo di aver messo le basi adatte per permettere una vita duratura al nostro software senza che esso subisca grossi cambiamenti nella struttura e nella logica di funzionamento.
\section{Obiettivi da migliorare}
Pur avendo assolto a gran parte degli obiettivi esistono alcuni margini di miglioramento. Il primo tra tutti una migliore gestione del vettore GPRS in quanto non molto usato nel nostro sviluppo a causa di una scarsa collaborazione da parte delle case produttrici nell'implementare i normali protocolli TCP/IP anche sulle connessioni GPRS.\\
Per quanto riguarda la telegestione il software pur essendo funzionante ed utilizzabile soffre ancora molto di problemi di stabilità dovuti principalmente a due fattori, il primo è che normalmente i protocolli che noi utilizziamo per la telegestione sono pensati per connessioni in rete locale e quindi con tempistiche diverse rispetto ad una gestione remota, in secondo luogo il bilanciamento tra intervallo di polling e la sensazione dell'operatore non è pienamente soddisfacente, infatti, attualmente il polling è molto stretto per far si che l'operatore non abbia la sensazione che il sistema sia bloccato tuttavia questo porta alcune volte a sovraccaricare le centrali di richieste; l'aspetto contrario è che aumentando l'intervallo del ciclo di polling l'operatore tende ad avere l'impressione che la centrale non risponda. Questo bilanciamento è molto difficile da ottenere in quanto la differenza tra i due aspetti è di pochi millisecondi.\\
Infine l'ultimo aspetto che si potrebbe migliorare ma che richiederebbe un notevole lavoro sia di progettazione che di adattamento del software è la base di dati, la quale è stata pensata e realizzata in un periodo in cui le esigenze e le specifiche erano differenti, oggi ci troviamo ad avere tabelle che non hanno più senso di esistere e altre create al solo scopo di soddisfare necessità imminenti ma senza tener conto del resto della base di dati.
\subsection{Sviluppi futuri}
Per quanto riguarda ciò che può essere realizzato gli obiettivi sono molti. Primo tra tutti dismettere il resto del vecchio software in modo da avere una conoscenza completa di tutto il software in produzione, non avere software proprietario Cobra, rendere completamente indipendenti i diversi moduli in modo da poter attivare e disattivare le diverse componenti a piacere.\\
Per quanto riguarda l'architettura delle macchine un possibile sviluppo è quello di utilizzare applicativi più aggiornati come l'ultima versione del JBoss o macchine di produzione aggiornabili e controllabili come versioni di Ubuntu con supporto a lungo termine con monitoraggio tramite Landscape.\\
Per quanto riguarda le nuove funzionalità da introdurre nel software invece esse sono diverse prima tra tutti la possibilità di integrare la videosorveglianza all'interno del software E-Pro ma accedendo direttamente all'evento d'allarme. La seconda funzionalità è l'integrazione del centralino con il resto del software questo permetterebbe di automatizzare alcune funzioni, come la chiamata diretta o quella automatizzata.