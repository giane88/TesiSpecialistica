\chapter{Introduzione}
\label{Introduzione}
\thispagestyle{empty}


\noindent La diffusione di connessioni a banda larga, il progressivo abbandono di reti telefoniche convenzionali e il passaggio su linee telefoniche VoIP hanno costretto le vigilanze private a trovare nuovi meccanismi di comunicazione verso gli apparati remoti di sicurezza da loro gestiti. Inoltre la richiesta di tempi di intervento più brevi e la necessità di un contatto minimo con il cliente richiedono strumenti di controllo e verifica immediati e di facile utilizzo.

\section{Inquadramento generale}
Questa tesi è stata sviluppata in collaborazione con \emph{LIS S.p.a.}, vigilanza privata che si distingue per i tempi di intervento ridotti e la possibilità di gestione degli impianti da remoto. Queste caratteristiche distinguono \emph{LIS} già dai primi anni di attività quando ancora la ricezione degli allarmi e la tele-gestione avveniva tramite linee telefoniche tradizionali.\\
Negli ultimi anni, tuttavia, la scomparsa delle linee telefoniche tradizionali in favore di quelle VoIP e di connessioni in fibra ottica hanno creato diversi problemi alla normale ricezione degli allarmi e ai meccanismi di telegestione. Si è deciso perciò di effettuare un aggiornamento del sistema esistente in modo da permettere a LIS di ricevere gli allarmi da linee mobili o tramite connessioni ADSL permettendo così una ricezione quasi istantanea della segnalazione di allarme e di conseguenza una gestione immediata dell'eventuale situazione di emergenza. Oltre alla ricezione degli allarmi un altro punto sul quale ci focalizzeremo è quello della gestione degli impianti tramite connessioni a banda larga o linee telefoniche mobili.\\
Prima di addentrarci nello specifico dobbiamo capire come lavora una vigilanza privata. Possiamo distinguere due operazioni principali che una vigilanza privata svolge, la prima è il lavoro di ricezione e verifica delle segnalazioni d'allarme provenienti dalle varie centrali di sicurezza e gli operatori, tramite l'ausilio di immagini provenienti da eventuali sistemi di videosorveglianza, valutano e gestiscono i vari eventi.La seconda funzione è quella di gestire gli impianti come ad esempio l'inserimento delle centrali antintrusione o l'esclusione di sensori guasti.\\
\section{Breve descrizione del lavoro}
Questa tesi ha lo scopo di progettare e realizzare un sistema unificato di ricezione allarmi e telegestione. In particolare la ricezione allarmi deve avvenire tramite diversi canali di comunicazione come SMS, GPRS, TCP/IP. La telegestione, invece, deve poter essere eseguita direttamente dal software già utilizzato da LIS.\\
In particolare ci siamo occupati dell'integrazione di protocolli per la ricezione degli allarmi analizzandoli ed implementandoli in un nuovo software che è stato affiancato al resto dei software già esistente. Lo stesso meccanismo è stato adottato per la realizzazione del modulo che gestisce la telegestione; tuttavia per rendere l'utilizzo di questo modulo performante si è deciso di effettuare una connessione diretta al software che gestisce l'interfaccia grafica.\\
Tutte queste funzioni sono state progettate e realizzate pensando ad una struttura che potesse facilitare successive integrazioni senza dover riscrivere interamente il software che si occupa di tali funzioni.


\section{Struttura della tesi}
La tesi è strutturata nel seguente modo:
\begin{itemize}
	\item Nel \chaptername~\ref{capitolo2} si inquadra l'ambito della tesi e si mostrano le principali alternative al nostro software.
	\item Nel \chaptername~\ref{capitolo3} si descrivono quali sono gli obiettivi che vogliamo ottenere e gli eventuali vincoli di progetto.
	\item Nel \chaptername~\ref{capitolo4} si illustrano i protocolli adibiti alla ricezione degli allarmi sui vettori TCP/IP e GPRS, inoltre si mostra la realizzazione del software necessario a ricevere queste segnalazioni.
	\item Nel \chaptername~\ref{capitolo5} si descrivono in modo generico alcuni protocolli che permettono la telegestione delle centrali e come essi siano stati integrati in modo da permettere la telegestione da E-Pro
	\item Nel \chaptername~\ref{capitolo6} si mostra il meccanismo che permette la comunicazione tra il software di telegestione descritto nel capitolo precedente ed il vecchio software E-Pro.
	\item Nel \chaptername~\ref{capitolo7} si illustra il processo di testing e validazione utilizzato in azienda; inoltre si esegue una piccola analisi dei risultati ottenuti.
	\item Nel \chaptername~conclusivo si riassumono gli scopi di questa tesi e si analizzano come essi siano stati portati a termine e quali invece possono essere migliorati.
\end{itemize}
Nell'appendice A si riportano i listati dei metodi o delle classi principali.
