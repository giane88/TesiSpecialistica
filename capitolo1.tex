\chapter{Introduzione}
\label{Introduzione}
\thispagestyle{empty}


\noindent La diffusione di connessioni a banda larga, il progressivo abbandono di reti telefoniche convenzionali e il passaggio su linee telefoniche VoIP hanno costretto le vigilanze private a trovare nuovi meccanismi di comunicazione verso gli apparati remoti di sicurezza da loro gestiti. Inoltre la richiesta di tempi di intervento più brevi e contatto con il cliente solo quando è indispensabile richiedono strumenti di controllo e verifica immediati e di facile utilizzo.

\section{Inquadramento generale}
Questa tesi è stata sviluppata in collaborazione con \emph{LIS s.r.l.}, vigilanza privata che si distingue per i tempi di intervento ridotti e la possibilità di gestione degli impianti da remoto. Queste caratteristiche distinguono \emph{LIS} già dai primi anni di attività quando ancora la ricezione degli allarmi e la tele-gestione avveniva tramite linee telefoniche tradizionali.\\
Negli ultimi anni tuttavia ci siamo trovati in difficoltà in quanto molte delle linee telefoniche tradizionali stanno scomparendo sostituite da fibre ottiche e linee VoIP questo impedisce la normale gestione. Si è deciso perciò di effettuare un aggiornamento del sistema in modo da permettere a \emph{LIS} di gestire gli impianti tramite le connessioni a banda larga o tramite linee telefoniche mobili. Oltre alla gestione degli impianti un altro punto sul quale ci focalizzeremo è quello della ricezione degli allarmi in modo tale da permettere una ricezione quasi istantanea della segnalazione di allarme e quindi una gestione immediata dell'eventuale situazione di emergenza.\\

\section{Breve descrizione del lavoro}
Questa tesi si sviluppa nell'ambito della sicurezza privata. Prima di addentrarci nello specifico dobbiamo capire come lavora una vigilanza privata. Possiamo distinguere due operazioni principali che una vigilanza privata svolge, la prima è il lavoro di ricezione e verifica delle segnalazioni d'allarme provenienti dalle varie centrali di sicurezza e gli operatori, tramite l'ausilio di immagini provenienti da eventuali sistemi di videosorveglianza, valutano e gestiscono i vari eventi.\\
La seconda funzione è quella di gestire gli impianti come ad esempio l'inserimento delle centrali anti-intrusione o l'esclusione di sensori guasti.\\
In LIS era già presente un sistema di che permetteva all'operatore di gestire 

\section{Struttura della tesi}
La terza parte contiene la descrizione della struttura della tesi ed \`e organizzata nel modo seguente.
``La tesi \`e strutturata nel modo seguente.

Nella sezione due si mostra \dots

Nella sez. tre si illustra \dots

Nella sez. quattro si descrive \dots

Nelle conclusioni si riassumono gli scopi, le valutazioni di questi e le prospettive future \dots

Nell'appendice A si riporta \dots (Dopo ogni sezione o appendice ci vuole un punto).''

I titoli delle sezioni da 2 a M-1 sono indicativi, ma bisogna cercare di mantenere un significato equipollente nel caso si vogliano cambiare. Queste sezioni possono contenere eventuali sottosezioni.

%``Terence: "Mi fai un gelato anche a me? Lo vorrei di pistacchio" \\
%Bud: "Non ce l'ho il pistacchio. C'ho la vaniglia, cioccolato, fragola, limone e caffé"\\
%Terence: "Ah bene. Allora fammi un cono di vaniglia e di pistacchio"\\
%Bud: "No, non ce l'ho il pistacchio. C'ho la vaniglia, cioccolato, fragola, limone e caffé"\\
%Terence: "Ah, va bene. Allora vediamo un po', fammelo al cioccolato, tutto coperto di pistacchio"\\
%Bud: "Ehi, macché sei sordo? Ti ho detto che il pistacchio non ce l'ho!"\\
%Terence: "Ok ok, non c'è bisogno che t'arrabbi, no? Insomma, di che ce l'hai?"\\
%Bud: "Ce l'ho di vaniglia, cioccolato, fragola, limone e caffé!"\\
%Terence: "Ah, ho capito. Allora fammene uno misto: mettici la fragola, il cioccolato, la vaniglia, il limone e il caffé. Charlie, mi raccomando il pistacchio, eh"''}
