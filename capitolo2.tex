\chapter{La sicurezza privata}
\label{capitolo2}
\thispagestyle{empty}

\begin{quotation}
	\noindent\footnotesize\emph{\textquotedblleft Giuro di osservare lealmente le leggi e le altre disposizioni vigenti nel territorio della Repubblica e di adempiere le funzioni affidatemi con coscienza e diligenza, nel rispetto dei diritti dei cittadini.\textquotedblright}
	\flushright{Giuramento di una guardia particolare giurata}
\end{quotation}
La sicurezza (dal latino "sine cura": senza preoccupazione) può essere definita come la "conoscenza che l'evoluzione di un sistema non produrrà stati indesiderati". In termini più semplici è: sapere che quello che faremo non provocherà dei danni.\cite{wiki:sicurezza}
\section{Le vigilanze private}
La vigilanza privata è l'attività, posta in essere da persone o da enti di coloro che operano nel campo della sicurezza privata, a tutela di persone, beni e/o enti pubblici o privati \cite{wiki:vigilanza}.\\
Le vigilanze private sono aziende che si occupano della protezione di persone e di beni mobili ed immobili, esse derivano direttamente dalle milizie cittadine del medioevo che, in tempo di pace svolgevano il compito di controllare e garantire la sicurezza dei cittadini durante la notte, nelle fiere e nei mercati.\\
Oggi le vigilanze private si occupano di diversi aspetti della sicurezza tramite l'utilizzo di tecnologie all'avanguardia. Tra queste attività troviamo:
\begin{description}
	\item[Piantonamento:] questo tipo di attività consiste nel presidio fisso da parte di una o più guardie particolari giurate (GPG) dotate di protezione anti proiettile e solitamente armate, esse sono collegate in modo costante con una centrale operativa. Solitamente tale attività viene svolta presso istituti di credito e enti pubblici. Possiamo distinguere tra piantonamenti diurni, piantonamenti notturni o piantonamenti per brevi periodi. Tale attività viene svolta in quei luoghi nei quali esiste un pericolo costante.
	\item[Servizio ispettivo:] questa attività consiste nell'ispezione saltuaria di alcune zone come piccole imprese, locali e aree circoscritte. Solto principalmente durante le ore notturne consiste in una visita della zona e nell'esame degli ingressi, degli infissi e del perimetro. Se la GPG durante l'ispezione nota delle anomalie provvede a contattare la centrale operativa che effettuerà gli opportuni controlli ed ad avvisare eventualmente le forze dell'ordine.
	\item[Trasporto valori:] in questo caso si tratta di un servizio di scorta effettuato da personale armato e dotato di protezioni antiproiettile ed effettuato tramite l'ausilio di mezzi blindati.
	\item[Sala conta:] questa attività è destinata soprattutto agli istituti di credito e ai centri commerciali. Il denaro viene prelevato dalla sede del cliente e prima di essere custodito nel \emph{caveau} dell'istituto di vigilanza vine ricontato trattato e confezionato secondo precise istruzioni.
	\item[Localizzazione satellitare:] tramite il sistema GPS è possibile localizzare a distanza un mezzo, inoltre è possibile effettuare alcune operazioni per gestire il mezzo in tempo reale. Tale servizio è rivolto soprattutto ai possessori di auto di valore, ad aziende di trasporto, ai mezzi blindati, e a chiunque abbia necessità di tenere sotto controllo la propria flotta di veicoli. Tale servizio è possibile grazie ad un apparecchio dotato di ricevitore GPS e di un interfaccia GSM o UMTS per la comunicazione dei dati.
	\item[Teleallarme:] questo servizio consiste nell'installazione di un sistema antitrusione in abbinata ad un sistema di teleallarme dove è necessario, collegati alla centrale operativa in modo da ricevere le eventuali segnalazioni di allarme e gestirle di conseguenza.
	\item[Telesoccorso:] molto simile al teleallarme ma questa volta la periferica invia le segnalazioni di allarme alla centrale solo nel caso in cui la persona prema un pulsante di allarme e non in modo automatico.
	\item[Videosorveglianza:] sistema complementare a quello di teleallarme o di telesoccorso avviene tramite l'utilizzo di telecamere collegate con la centrale operativa dell'istituto di vigilanza. Tale meccanismo permette di valutare la reale presenza di eventuali pericoli e di guidare i controlli.
\end{description}
\subsection{La vigilanza di LIS}
\section{Le tecnologie di LIS}
\section{Cosa offre il mercato}